\chapter{Дополнительная информация по работе с~Simics} \label{chap:append01}

В данное приложение включены сведения о различных приёмах, используемых при ежедневном использовании Simics, не описанные в главах, посвящённых индивидуальным лабораторным работам. Данный материал не заменяет необходимость ознакомления с официальной документацией Simics, а лишь подчёркивает ключевые моменты в ней.

\section{Обновление workspace}

Для получения последних исправлений ошибок в моделях необходимо использовать самую свежую версию базового пакета Simics из  установленных на системе. Номера доступных версий можно определить по именам существующих директорий (по умолчанию в \texttt{/opt/simics}). В дальнейших примерах последней версией будет считаться \textbf{4.6.32}, при этом 4.6 --- это основная версия, а последняя цифра --- номер минорной версии обновления, котороый будет использован ниже.

Каждая копия workspace характеризуется \emph{комбинацией} версий пакетов, в ней используемых. Версия пакета Simics Base (\textnumero 1000) определяет настройки версий остальных пакетов, установленных одновременно с ним. Для того, чтобы увидеть список установленных пакетов и их версии, используйте ключ \texttt{-v} при запуске Simics:

\begin{lstlisting}
$ ./simics -v
Simics Base                                        1000    4.6.32    (4051)
Model Library: Intel Core i7 with X58 and ICH10    2075    4.6.21    (4051)
Model Builder                                      1010    4.6.14    (4051)
Extension Builder                                  1012    4.6.6     (4042)
\end{lstlisting}

Также версию Simics можно узнать из командной строки любой уже запущенной симуляции c помощью команды \texttt{version}:

\begin{lstlisting}
simics> version
Installed Products:

Model Builder
Extension Builder
Model Library: Intel Core i7 with X58 and ICH10

Installed Packages:

Package                  Nbr  Version  Build  Sources
=====================================================
Simics-Base             1000  4.6.32    4145       No
Model-Builder           1010  4.6.49    4146       No
Extension-Builder       1012  4.6.19    4141       No
x86-Core-i7-X58-ICH10   2075  4.6.59    4146       No
\end{lstlisting}

В примере сверху базовый пакет имеет версию 4.6.32. Обновления пакетов могут периодически устанавливаться в вашей системе для исправления ошибок в предоставляемых моделей. Однако уже созданные workspace будут по-прежнему использовать старые версии, если для них не выполнить процедуру обновления. 

Для обновления workspace, как и для его создания, используется программа \texttt{workspace-setup}, находящаяся внутри новой версии базового пакета (версии 4.6.\textbf{<minor>}).

\begin{lstlisting}
$ /opt/simics/simics-4.6/simics-4.6.<minor>/bin/workspace-setup
Workspace updated successfully
\end{lstlisting}

\section{Список часто используемых команд Simics}

\begin{center}
\small
\begin{tabularx}{\textwidth}{Xlp{0.4\textwidth}}
%\begin{longtable}{|p{0.3\textwidth}|l|p{0.3\textwidth}|}\
\textbf{Команда}                          & \textbf{Синонимы}  & \textbf{Выполняемая функция}\\\hline
\texttt{help <topic>}                     & \texttt{man}       & Справка по команде, классу или слову topic\\
\texttt{win-help}                         &                    & Открыть окно индексируемой справки \\
\texttt{continue}                         & \texttt{c, r, run} & Начать или продолжить симуляцию\\
\texttt{stop}                             &                    & Остановить симуляцию\\
\texttt{step-cycle [count]}               & \texttt{sc}        & Исполнить count циклов, печатаю следующую инструкцию\\
\texttt{exit}                             & \texttt{quit, q}   & Выйти из симулятора\\
\texttt{run-command-file <script.simics>} &                    & Выполнить скрипт Simics\\
\texttt{pregs [-all]}                     &                    & Распечатать содержимое регистров текущего процессора\\
\texttt{print-time [-all]}                & \texttt{ptime}     & Вывести значение виртуального времени процессора\\
\texttt{win-control}                      &                    & Открыть окно \textbf{Simics Control}\\
\texttt{\%<register name>}                & \texttt{read-reg}  & Прочитать содержимое регистра текущего процессора \\
\texttt{\%<register name> = <val>}        & \texttt{write-reg} & Записать значение в регистр текущего процессора \\
\texttt{output-radix <10|16>}             &                    & Изменить основание используемой для вывода чисел системы счисления\\
\texttt{break <address>}                  &                    & Поставить точку останова по адресу \\
\texttt{delete [id]}                      &                    & Удалить точку останова по её номеру \\
\end{tabularx}

% \end{longtable}
\end{center}


