\chapter*{Предисловие}\label{chap:contrib}

В настоящем практикуме описываются лабораторные и практические работы по курсу <<Основы программного моделирования ЭВМ>>, проводящиеся в Московском физико-техническом институте. Эта книга дополняет практическими аспектами программирования и использования технологий теоретические основы, изложенные в учебнике: 

Основы программного моделирования ЭВМ: Учебное пособие / Г. Речистов, Е. Юлюгин, А. Иванов, П. Шишпор, Н. Щелкунов, Д. Гаврилов. — 2-е изд., испр. и доп. — Издательство МФТИ, окт. 2013. — ISBN 978-5-7417-0444-8.

Авторы прикладывают усилия для того, чтобы поддерживать все учебные материалы в актуальном состоянии. Самую свежую версию данного документа вы можете получить на сайте \url{http://atakua.doesntexist.org/wordpress/simulation-course-russian/}.

Если вы обнаружили опечатку, стилистическую, фактическую ошибку, которые, более чем вероятно, встречаются в тексте, имеете замечания по содержанию или предложения по тому, как можно улучшить данный материал, то просим сообщить об этом по e-mail \url{grigory.rechistov@phystech.edu} --- нам очень важно ваше мнение!

Отметим, что текст данной работы постоянно обновляется, и поэтому в версиях, имеющих в своём номере пометку «бета» ($\beta$), могут присутствовать незаконченные места, которые обозначаются символом \todo.

