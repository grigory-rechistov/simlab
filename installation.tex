\chapter{Установка и лицензирование Simics} \label{chap:installation-notes}

В данное приложение включена информация по лицензированию и установке Simics в учебной лаборатории. Наиболее полная информация по данному вопросу содержится в документе «Simics Installation Guide», который идёт в поставке с пакетами Simics.

Приводимые ниже инструкции были собраны для Simics версии 4.6 для хозяйской системы Linux 64 бит, рекомендуемой для всех пользователей. Для ОС Linux 32 бит инструкции изменяются незначительно; для ОС Windows они применимы после учёта особенностей графического процесса инсталляции.

\section{Академическая программа Wind River Simics}

\todo

\section{Получение и продление лицензии}

\todo

\subsection{Условия предоставления}

Использование Simics по академической программе должно строго соответствовать условиям соглашения, т.е. быть ограничено учебной и/или некоммерческой научно-исследовательской деятельностью. В случае возникновения необходимости проведения коммерческих исследований или разработок необходимо обратиться к представителям Wind River для получения нового соглашения и типа лицензии.

Держатель лицензии обязан донести эту информацию до всех пользователей инсталляции и контролировать выполнение ими условий соглашения, в том числе с помощью административных и технических мер.

\todo ограничения файервола
\todo Порты

\section{Установка пакетов Simics}

Каждый пакет --- это файл с именем вида \todo{}. 

Ключи шифрования получите у спонсора вашей академической программы.

\begin{enumerate}
    \item Разархивируйте все пакеты во временую директорию:
    
    
    \item Запустите скрипт установки
    
    
    \item Ответьте на вопросы
    
    \item При запросе файла с лицензией пропустите шаг
    
    \item 
\end{enumerate}

Проверьте указанную при установке директорию --- в ней должны находиться подпапки с файлами Simics.

\subsection{Получение lmhostid}

Для получения файла лицензии необходимо сгенерировать и передать число, так называемый lmhostid

\paragraph{Об именовании сетевых интерфейсов.} На момент написания данного материала утилиты из состава Simics не поддерживали получение корректного lmhostid на системах, использующих схему «стабильного именования» сетевых интерфейсов. Вместо традиционных для Linux имён \texttt{eth0}, \texttt{eth1} и т.д. сетевым картам выдаются имена, зависящие от производителя и физического расположения в системе.

Для обеспечения \todo

\begin{enumerate}
    \item Установите пакет \texttt{lsb-core} на системе. Для Debian и Ubuntu это выполняется командой 
    
    \texttt{\# apt-get install lsb-core}
    
    
    \item Установите пакеты Simics. При запросе лицензии не указывайте ничего.
    
    \item Для получения lmhostid на сервере, \textit{который будет использоваться для запуска демона лицензий}, выполните команду:
\end{enumerate}


Возможные решения:



\todo{Команда}

\subsection{Заполнение заявки}




\section{Настройка сервера лицензий}


\subsection{Файл лицензии}

Получаемый от производителя файл лицензии --- это текстовый документ, содержащий информацию о сроке действия, ограничениях количества одновременно запускаемых копий и поддерживаемых расширениях приложения. Пример содержимого для начала этого файла:

\begin{lstlisting}
# Simics 4.6 licence for the Simics Academic Program
#
# University:        Moscow Institute of Physics and Technology
# Contact:           academic.contact@university.edu
# Sponsor:           sponsor.contact@sponsor.com
# Licensing Contact: licencing.contact@licencer.com
#
SERVER  lic.university.edu lmhostid
VENDOR simics /home/Incoming/simics-4.6/simics-4.6.100/flexnet/linux64/bin
#
FEATURE simics simics 4.6 28-feb-2014 50 BD47D265FA68 \
        VENDOR_STRING=intel;academic HOSTID=ANY BORROW TS_OK \
        SIGN="0441 6AFA 450C BDBE E4D7 E125 1042 EEFF 04B5 767A ABCD \
        5088 80DB D912 292E 4FD5 22DD 22D0 D55F 5B25 4818"
<...>
\end{lstlisting}

Не изменяйте никаких строчек этого файла, кроме имени сервера лицензий. Сохраните копию файла в надёжном месте. Запишите дату окончания действия лицензии для последующей своевременной инициации процедуры обновления.

\subsection{Запуск сервера}

\subsection{Проверка работоспособности}



\section{Расположение файлов и сервера лицензий при подключении по сети}

По умолчанию все файлы Simics размещаются в директории \texttt{/opt/simics}. Если необходимо обеспечить запуск симулятора на нескольких компьютерах, подсоединённых по сети, рекомендуется разместить эти файлы установки в файловой системе, доступной по сети, например, по протоколам NFS или CIFS. Таким образом, клиентские машины смогут переиспользовать структуру инсталляции без необходимости её копирования на локальные диски, что упростит поддержку и обновления. Настройка сетевой файловой системы выходит за рамки данного руководства; необходимую информацию можно найти, например, в~\cite{nfs}.

\subsection{Финальный вид инсталляции}

На рис.~\ref{fig:install-overview} приведена рекомендуемая схема соединения систем и расположения служб для работы Simics на всех компьютерах учебного класса или лаборатории. В данном примере сервер для запуска демона лицензий отделён от сервера общих файлов; на практике они могут быть одной и той же системой.

\begin{figure}[htbp]
\centering
\begin{tikzpicture}[>=latex, font=\small]
    
    \node[draw] (lic-file) {simics-00AA-2014.lic};
    \node[draw, below = 0.25cm of lic-file] (lmgrd) {lmgrd демон};
    \node[above = 0.25cm of lic-file] (lic-host) {lic.university.edu};
    \node[draw, fit = (lic-file) (lmgrd) (lic-host)] (lic-server) {};    
    
    \node[draw, below = 2cm of lic-server] (installation) {/opt/simics/...};
    \node[draw, below = 0.25cm of installation] (nfs) {NFS демон};
    \node[above = 0.25cm of installation] (nfs-host) {nfs.university.edu};
    \node[draw, fit = (installation) (nfs) (nfs-host)] (nfs-server) {};
    
    \node[draw, right = 3cm of lic-file] (simics01) {Симуляция};
    \node[draw, below = 0.25cm of simics01] (lab01-client) {Клиент Simics};
    \node[above = 0.25cm of simics01] (lab01-host) {lab01.university.edu};
    \node[draw, fit = (simics01) (lab01-client) (lab01-host)] (lab01) {};

    \node[draw, below = 1.5cm of lab01] (simics02) {Симуляция};
    \node[draw, below = 0.25cm of simics02] (lab02-client) {Клиент Simics};
    \node[above = 0.25cm of simics02] (lab02-host) {lab02.university.edu};
    \node[draw, fit = (simics02) (lab02-client) (lab02-host)] (lab02) {};

    \node[draw, below = 1.5cm of lab02] (simics03) {Симуляция};
    \node[draw, below = 0.25cm of simics03] (lab03-client) {Клиент Simics};
    \node[above = 0.25cm of simics03] (lab03-host) {lab03.university.edu};
    \node[draw, fit = (simics03) (lab03-client) (lab03-host)] (lab03) {};
    
    \draw[->] (lmgrd.east) -- (lab01-client.west);
    \draw[->] (lmgrd.east) -- (lab02-client.west);
    \draw[->] (lmgrd.east) -- (lab03-client.west);
    
    \coordinate[right = 1cm of installation]  (midpoint);
    \draw[<-] (installation) -| (midpoint);
    \draw[->] (midpoint) |- (simics01);
    \draw[->] (midpoint) |- (simics02);
    \draw[->] (midpoint) |- (simics03);
    
\end{tikzpicture}
\caption{Расположение и функции узлов инсталляции Simics}\label{fig:install-overview}
\end{figure}


\subsection{Решение возникших проблем}

\begin{description}
    \item[Невозможно стартовать lmgrd]
    
    \item[Невозможно запустить Simics]
    
    \item[Нет подключения к демону лицензии с сервера]

    
    \item[Нет подключения к демону лицензии по сети] 
    
    \item[Демон лицензий не работает после перезагрузки сервера] 
    
    
    \item[Исчерпано число лицензий]
    
\end{description}

\section{Обновление пакетов инсталляции}

\todo

\iftoggle{webpaper}{
    \printbibliography[title={Литература}]
}{}

