% Compile with XeLaTeX, TeXLive 2013 or more recent
\documentclass{beamer}

% Base packages
\usepackage{fontspec}
\usepackage{xunicode}
\usepackage{xltxtra}

\usepackage{amsfonts}
\usepackage{amsmath}
\usepackage{longtable}
\usepackage{csquotes}
\usepackage{standalone}

% Setup fonts
\newfontfamily\russianfont{CMU Serif}
\setromanfont{CMU Serif}
\setsansfont{CMU Sans Serif}
\setmonofont{CMU Typewriter Text}

% Setup Russian hyphenation. NOTE: this declaration *must* come after fontspec's font declarations,
% or a mysterious (but harmless in other respects) error "Improper `at' size (0.0pt), replaced by 10pt." would appear.
\usepackage{polyglossia}
\defaultfontfeatures{Scale=MatchLowercase, Mapping=tex-text}

\setdefaultlanguage[spelling=modern]{russian} % for polyglossia
\setotherlanguage{english} % for polyglossia

% Vector drawings 
\usepackage{tikz}
\usetikzlibrary{shapes, calc, arrows, fit, positioning, decorations, patterns, decorations.pathreplacing, chains, snakes}
\usepackage[siunitx]{circuitikz}

% Be able to insert hyperlinks
\usepackage{hyperref}
\hypersetup{colorlinks=true, linkcolor=black, filecolor=black, citecolor=black, urlcolor=blue , pdfauthor=Grigory Rechistov <grigory.rechistov@phystech.edu>, pdftitle=Моделирование OpenRISC 1000 на Wind River Simics}
% \usepackage{url}

% Misc optional packages
\usepackage{underscore}
\usepackage{amsthm}

% A new command to mark not done places
\newcommand{\todo}[1][Напиши меня]{{\color{red}TODO\ #1}}

\title{Компьютерная симуляция }
% \subtitle{Курс «Программное моделирование вычислительных систем»}
\subject{Лекция}
\author[Григорий Речистов]{Григорий Речистов \\ \small{\href{mailto:grigory.rechistov@intel.com}{grigory.rechistov@intel.com}}}
\date{\today}
\pgfdeclareimage[height=0.5cm]{intel-logo}{../images/intel.png}
\logo{\pgfuseimage{intel-logo}}


\usetheme{Berlin}
\setbeamertemplate{navigation symbols}{}%remove navigation symbols

\begin{document}

\begin{frame}
\titlepage
\end{frame}

\begin{frame}
\tableofcontents
\end{frame} 

\section{Знакомство}

\begin{frame}{Обо мне}

\raggedleft \includegraphics[height=0.3\textheight]{./grigory-rechistov}

\begin{itemize}
\item Закончил МФТИ в 2010 г.
\item Защитил диссертацию к.т.н в 2013 г.
\item Работаю в Московском отделении Intel.
\item Интересы: симуляция, образование, спорт.
\end{itemize}
\end{frame} 

\begin{frame}{О вас}

Поднимите руки те, кто знаком с\pause
\begin{enumerate}
\item Языком Си.\pause
\item Linux.\pause
\item English.\pause
\item Архитектурой ЭВМ.\pause
\item Python.\pause
\item Make.\pause
\item SVN/Git.\pause
\item VirtualBox, Qemu, Bochs, VMWare.
\end{enumerate}

\end{frame}

\begin{frame}{Почему симуляция актуальна для \emph{вас}?}

\begin{enumerate}
\item Это интересно — как работают компьютеры внутри!
\item Помогает стать лучшим программистом — почему код работает именно так, а не иначе (даёт ответы на загадки необъяснимых падений, плохой производительности).
\item Знания востребованы работодателями — HPC, embedded, gaming, ОС \dots
\item Многие алгоритмы/идеи симуляции являются общими для всей области CS, в т.ч. компиляции, ОС, прикладного ПО.
\end{enumerate}

\end{frame} 

\section{Кратко о симуляции}

\begin{frame}{Место симуляции в области Computer Science}

\centering

\begin{tikzpicture}[>=latex, font=\small]

\begin{scope}[minimum width = 8cm, node distance = 0.2cm, inner sep=2pt]

\node[draw, ] (schematics) {Схемотехника};
\node[draw, above = of schematics] (digital) {Цифровая электроника};
\node[draw, above = of digital] (rtl-design) {Проектирование микросхем};
\node[draw, above = of rtl-design] (transactions) {Разработка стандартов передачи данных};
\node[draw, thick, above = of transactions] (simulation) {\bfseries Симуляция};
\node[draw, above = of simulation] (firmware) {Firmware};
\node[draw, above = of firmware] (oses) {Операционные системы};
\node[draw, above = of oses] (drivers) {Драйверы};
\node[draw, above = of drivers] (compilers) {Компиляторы};
\node[draw, above = of compilers] (software) {Прикладные программы};
\node[draw, above = of software] (algorithms) {Алгоритмы};

\end{scope}
\end{tikzpicture}

\end{frame}

\begin{frame}{Wind River\textsuperscript{\textregistered}~Simics}
\raggedleft \includegraphics[height=0.3\textheight]{./simics-logo}

\begin{itemize}
\item Симулятор виртуальных платформ.
\item Фреймворк для создания моделей новых цифровых устройств.
\item Позволяет моделировать: устройства, узлы, компьютеры, сети компьютеров.
\end{itemize}

\end{frame}

\begin{frame}{Что будет дальше}
\begin{itemize}
\item День 1: использование Simics.
\item День 2: как устроены модели процессоров и периферии.
\item День 3: написание модели процессора OpenRISC 1000.
\end{itemize}
\end{frame}

\begin{frame}{Цели занятий}
\begin{itemize}
\item Научиться пользоваться Simics как пользователи.\pause
\item Научиться пользоваться Simics как создатели новых моделей.\pause
\item Заглянуть в устройство компьютера.\pause
\item To have fun!
\end{itemize}

\end{frame}


\section{Литература}

\begin{frame}[allowframebreaks]{Литература}
\begin{thebibliography}{99}
    \bibitem{simbook} Программное моделирование вычислительных систем (лекции) \url{http://atakua.doesntexist.org/public/archive/simcourse/simulation-lectures-latest.pdf}
	
	\bibitem{practicum} Лабораторный практикум по программному моделированию.  \url{http://atakua.doesntexist.org/public/archive/simcourse/simulation-practicum-latest.pdf}

\end{thebibliography}
\end{frame}


\section{Конец}
% The final "thank you" frame 
\begin{frame}

{\huge{Спасибо за внимание!}\par}

\vfill

Слайды и материалы курса достапны по адресу \url{http://bit.ly/1mr9eCP} % http://atakua.doesntexist.org/wordpress 

\vfill

\tiny{\textit{Замечание}: все торговые марки и логотипы, использованные в данном материале, являются собственностью их владельцев. Представленная точка зрения отражает личное мнение автора.
%Материалы доступны по лицензии Creative Commons Attribution-ShareAlike (Атрибуция — С сохранением условий) 4.0 весь мир (в т.ч. Россия и др.). Чтобы ознакомиться с экземпляром этой лицензии, посетите \url{http://creativecommons.org/licenses/by-sa/4.0/}
}

\end{frame}

% \section{Резерв}

\end{document}


