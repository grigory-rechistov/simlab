% Compile with XeLaTeX, TeXLive 2013 or more recent
\documentclass{beamer}

% Base packages
\usepackage{fontspec}
\usepackage{xunicode}
\usepackage{xltxtra}

\usepackage{amsfonts}
\usepackage{amsmath}
\usepackage{longtable}
\usepackage{csquotes}
\usepackage{standalone}

% Setup Russian hyphenation
\usepackage{polyglossia}
\defaultfontfeatures{Scale=MatchLowercase, Mapping=tex-text}

% Setup fonts
\newfontfamily\russianfont{CMU Serif}
\setromanfont{CMU Serif}
\setsansfont{CMU Sans Serif}
\setmonofont{CMU Typewriter Text}

\setdefaultlanguage[spelling=modern]{russian} % for polyglossia
\setotherlanguage{english} % for polyglossia

% \usepackage{libertine}

% Vector drawings 
\usepackage{tikz}
\usetikzlibrary{shapes, calc, arrows, fit, positioning, decorations, patterns, decorations.pathreplacing, chains, snakes}

% Be able to insert hyperlinks
\usepackage{hyperref}
\hypersetup{colorlinks=true, linkcolor=black, filecolor=black, citecolor=black, urlcolor=blue , pdfauthor=Grigory Rechistov <grigory.rechistov@phystech.edu>, pdftitle=Моделирование OpenRISC 1000 на Wind River Simics}
% \usepackage{url}

% Misc optional packages
\usepackage{underscore}
\usepackage{amsthm}

% A new command to mark not done places
\newcommand{\todo}[1][Напиши меня]{{\color{red}TODO\ #1}}


\title{Компьютерная симуляция }
% \subtitle{Курс «Программное моделирование вычислительных систем»}
\subject{Лекция}
\author[Григорий Речистов]{Григорий Речистов \\ \small{\href{mailto:grigory.rechistov@intel.com}{grigory.rechistov@intel.com}}}
\date{\today}
\pgfdeclareimage[height=0.5cm]{intel-logo}{../images/intel.png}
\logo{\pgfuseimage{intel-logo}}


\typeout{Copyright 2014 Grigory Rechistov}

\usetheme{Berlin}
\setbeamertemplate{navigation symbols}{}%remove navigation symbols

\begin{document}

\begin{frame}
\titlepage
\end{frame}

\begin{frame}
\tableofcontents
\end{frame} 

\section{Обзор}

\begin{frame}{На прошлой лекции}
\begin{itemize}
\item Оптимистичные схемы
\item Откат состояния 
\item Time Warp и virtual time \pause
\item Вопрос: как выполнять вывод (\texttt{printf}) в оптимистичной симуляции?
\end{itemize}

\end{frame}

\section{Атомарные инструкции}

\begin{frame}{Атомарные инструкции}

\begin{tikzpicture}[>=latex]

\node[draw, circle] (core1) {Ядро 1};

\node[draw, circle, right = of core1] (core2) {Ядро 2};

\node[draw, circle, right = of core2] (core3) {Ядро 3};

\node[draw, circle, right = of core3] (core4) {Ядро 4};

\node[below = 2cm of core1] (c1) {};
\node[below = 2cm of core4] (c2) {};

\node[draw, fit = (c1) (c2) ] (shmem) {Общая память} ;

\path (core2.south) -- (core3.south) node[midway, yshift = -1cm] (lock) {\texttt{LOCK}};
\draw[thick,] (core2) |- (lock);
\draw[thick,->] (lock) -- (shmem);

\path (core1.south) |- (lock) coordinate[midway] (block1); %{X};
\path (core3.south) |- (lock) coordinate[midway] (block3); %{X};
\path (core4.south) |- (lock) coordinate[midway] (block4); %{X};

\draw[->] (core1.south) -- (block1);
\draw[->] (core3.south) -- (block3);
\draw[->] (core4.south) -- (block4);

\draw[dashed] (block1) -- (lock);
\draw[dashed] (lock)   -- (block3);
\draw[dashed] (block3) -- (block4);

\end{tikzpicture}

\begin{itemize}
    \item {Read--Modify--Write} для ячейки в памяти
    \item Средства реализации семафоров
    \item «Дорогие» для исполнения \pause
    \item {Вопрос}: нужны ли атомарные инструкции для однопроцессорных систем?
\end{itemize}

\end{frame}

\begin{frame}{Симуляция инструкций}

\begin{enumerate}
    \item Использование хозяйских инструкций
    \item Использование критических секций
    \item Использование транзакций
\end{enumerate}
\end{frame}

\section{Литература}

\begin{frame}[allowframebreaks]{Рекомендуемая литература}
\begin{thebibliography}{99}
    \bibitem{consensus-number} \textit{Maurice Herlihy}. “Wait-Free Synchronization” \url{http://cs.brown.edu/~mph/Herlihy91/p124-herlihy.pdf}
    \bibitem{coremu} \textit{Zhaoguo Wang et al.} COREMU: a Scalable and Portable Parallel Full-System Emulator \url{http://ppi.fudan.edu.cn/_media/publications\%3Bcoremu-ppopp11.pdf},
    \bibitem{consistency-report} \textit{Kourosh Gharachorloo} Memory Consistency Models for Shared-Memory Multiprocessors. \url{http://infolab.stanford.edu/pub/cstr/reports/csl/tr/95/685/CSL-TR-95-685.pdf}
    \bibitem{whymb} \textit{Paul E. McKenney} Memory Barriers: a Hardware View for Software Hackers \url{http://citeseerx.ist.psu.edu/viewdoc/summary?doi=10.1.1.152.5245}
    \bibitem{pqemu} \textit{Jiun-Hung Ding et al.} PQEMU: A Parallel System Emulator Based on QEMU \url{http://dx.doi.org/10.1109/ICPADS.2011.102}
    \bibitem{itanium-mem-order} \textit{Intel Corporation} A Formal Specification of {Intel}® {Itanium}® Processor Family Memory Ordering
\end{thebibliography}
\end{frame}


\section{Конец}
% The final "thank you" frame 
\begin{frame}

{\huge{Спасибо за внимание!}\par}

\vfill

\vfill

\tiny{\textit{Замечание}: все торговые марки и логотипы, использованные в данном материале, являются собственностью их владельцев. Представленная точка зрения отражает личное мнение автора.}

\end{frame}

% \section{Резерв}

\end{document}
