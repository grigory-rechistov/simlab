\chapter{Моделирование процессора архитектуры OpenRISC 1000}\label{chap:lab06}

Данная практическая Ну работа посвящена реализации модели компьютерной платформы, основанной на спецификации OpenRISC 1000~\cite{or1k-spec}. Модель строится на основе API Simics и оформляется как набор модулей и сценариев для данного симулятора.

\section{Спецификации OpenRISC 1000}

OpenRISC 1000 --- дизайн процессора, построенный по философии Open Source~\cite{open-source-definition-ru} как для программного кода, так и для аппаратуры. Спецификация определяет 32- или 64-битный RISC-процессор общего назначения, с пятью стадиями конвеера, необязательной поддержкой MMU (\abbr memory management unit), кэшей, управлением энергопотреблением и др. Важная для данной лабораторной работы черат спецификации OpenRISC 1000 --- это опциональность многих частей функциональности, что позволяет ограничиваться только минимально необходимыми для конкретной реализации элементами.

Существуют реализации OpenRISC 1000 в виде программных моделей, спецификаций для FPGA и описаний RTL на языке Verilog. Программная поддержка существует со стороны компиляторов GCC и LLVM, адаптированных библиотек LibC и GNU toolchain.

\subsection{Этапы проекта}

Для создания рабочей функциональной модели ЦПУ требуется спецификация на следующие его элементы: архитектурное состояние (число и типы регистров), набор инструкций (кодировка и семантика), а также интерфейсы к внешним элементам системы (память и периферийные устройства). Детали, относящиеся к особенностям организации конвеера, длительностям работы отдельных инструкций, устройству кэшей и т.п. не являются необходимыми на данном уровне точности моделирования.

\section{Краткое введение в архитектуру OpenRISC 1000}

\subsection{Набор регистров}

Классификация регистров OpenRISC 1000 (рис.~\ref{fig:or1k-regs}), требующих реализации в ходе работы.
\begin{itemize}
    \item 32 регистра общего назначения шириной в 64 бита, доступные из пользовательского и супервизорного режимов: \texttt{R0} -- \texttt{R31}. Регистр R0 всегда должен содержать ноль.
    \item Специфичные для модулей регистры (\abbr special purpose registers, SPR). В случае, если некоторая опциональная часть спецификации реализована, с ней могут идти дополнительные регистры для индикации статуса и управления состоянием. Каждый из них имеет собственное имя и функциональность. SPR разделены по группам. Группа 0 --- регистры супервизора шириной 32 бита. Некоторые из них могут быть доступны на чтение из пользовательского режима.
\end{itemize}

\begin{figure}[htbp]
\centering

\bytefieldsetup{bitwidth=0.07cm, endianness = big}
\begin{bytefield}[]{32}
    \bitheader{31, 0} \\
    \bitbox{32}{PC} \\
\end{bytefield}

\bytefieldsetup{bitwidth=0.07cm, endianness = big}
\begin{bytefield}[]{64}
    \bitheader{63, 0} \\
    \bitbox{64}{GPR0} \\
    \bitbox{64}{\dots} \\
    \bitbox{64}{GPR31} \\
\end{bytefield}

\begin{bytefield}[]{32}
    \bitheader{31, 0} \\
    \bitbox{32}{VR} \\
    \bitbox{32}{UPR} \\
    \bitbox{32}{CPUCFGR} \\
    \bitbox{32}{\dots} \\
    \bitbox{32}{ESR15} \\
\end{bytefield}


% \begin{tikzpicture}[>=latex, node distance=0.5cm, font=\small]
%     \node[draw] (gpr0) {GPR0};
%     \node[draw] (gpr31) {GPR31};
% 
% %     \node[draw, below = of cpu.south west, anchor=north west] (graph) {\textsc{graphics}: graph16};
%     
% \end{tikzpicture}
\caption{Регистры OpenRISC 1000}\label{fig:or1k-regs}
\end{figure}

\subsection{Набор инструкций}

Набор инструкций состоит из нескольких классов (рис.~\ref{fig:or1k-isa}), соответствующих различным задачам и типам обрабатываемых данных.

\begin{figure}[htbp]
\centering
\begin{tikzpicture}[font=\small, node distance = 0.5cm]
    \node[draw, ellipse] (ORBIS32) {ORBIS32};
    \node[draw, ellipse, right = of ORBIS32] (ORBIS64) {ORBIS64};
    \node[draw, ellipse, below = of ORBIS32] (ORFPX32) {ORFPX32};
    \node[draw, ellipse, below = of ORBIS64] (ORFPX64) {ORFPX64};
    \node[draw, ellipse, below = of ORFPX64] (ORVDX64) {ORVDX64};
    \node[draw, ellipse, fit = (ORBIS32) (ORBIS64) (ORFPX32) (ORFPX64) (ORVDX64)] {};
\end{tikzpicture}
\caption{Классы инструкций набора команд OpenRISC 1000}\label{fig:or1k-isa}
\end{figure}

\begin{itemize}
    \item ORBIS32 --- 32-битные команды общего назначения; единственный класс, обязательный для реализации;
    \item ORBIS64 --- 64-битные команды общего назначения;
    \item ORFPX32 --- 32-битные команды для работы над числами с плавающей запятой;
    \item ORFPX64 --- 64-битные команды для работы над числами с плавающей запятой;
    \item ORVDX64 --- команды для операции над векторами данных с элементами шириной 8, 16, 32 или 64 бита;
    \item Частные команды, специфичные для конкретной модели.
\end{itemize}

Ширина отдельных инструкций фиксирована и равна 32 битам. ORBIS32 состоит из следующих подклассов команд:

\begin{itemize}
    \item арифметические операции;
    \item базовые инструкции для обработки сигналов (\abbr digital signal processing, DSP);
    \item загрузка и запись данных из памяти;
    \item управление порядком исполнения;
    \item прочие.
\end{itemize}

\section{Кодовая база Simics для создания моделей процессоров}

Основа модели ЦПУ --- код \texttt{sample-risc}, модифицированный для нужд учебного проекта:
\begin{itemize}
    \item убрана мноноядерность;
    \item декодер упрощён до одной инструкции.
\end{itemize}

Основа модели таймера --- sample-device.

Конфигурация платформы (ЦПУ, память, опционально таймер) --- основана на сценарии \texttt{queues-in-cosimulator.simics}.

\section{Тестирование моделей}

Для тестирования корректности моделирования отдельных инструкций применяется юнит-тестирование на основе фреймворка test-runner. Для проверки последовательностей команд код гостевой код может быть написан вручную в машинных кодах или же с помощью GNU toolchain~\cite{or1k-toolchain}.


\section{Порядок работы}

\begin{enumerate}
\item Подготовить workspace Simics.
\item Собрать код sample-risc.
\item Реализовать набор регистров базового архитектурного состояния.
\item Реализовать декодер команд.
\item Интегрировать модель openrisc в сценарий платформы.
\item Реализовать модель таймера.
\item Интегрировать модель таймера в сценарий платформы.

\end{enumerate}

\subsection{Дополнительные шаги}

\begin{itemize}
    \item Реализовать набор инструкций ORBIS64.
    \item Реализовать функциональность TLB и/или MMU.
    \item Реализовать устройство PIC.
    \item Реализовать модель кэша данных.
    \item Реализовать набор инструций ORFPX32.
\end{itemize}

\iftoggle{webpaper}{
    \printbibliography[title={Список литературы к занятию}]
}{}


% День 1
% 1.	[Лекция, 3 часа] Программное моделирование для задач совместной разработки аппаратуры и программ. Другие применения программных моделей. История использования. Симулятор Wind River Simics. Принципы работы, базовые понятия.
% 2.	[Практикум, 5 часов] Знакомство с Simics: создание проекта, запуск моделирования, управление и инспектирование состояния симуляции. Сборка моделей шаблонных устройств.
% День 2 
% 1.	[Лекция, 2 часа] Моделирование центральных процессоров с помощью интерпретации. Моделирование периферийных устройств с помощью симуляции дискретных событий.
% 2.	[Лекция, 1 час] Архитектура OpenRISC 1000. 
% 3.	[Практикум, 5 часов] Создание модели процессора OpenRISC 1000 с использованием фреймворка Simics Model Builder: архитектурное состояние, набор инструкций ORBIS32.
% День 3
% 1.	[Практикум, 8 часов] Продолжение реализации основной и опциональной функциональности модели OpenRISC 1000.
% а.	Набор инструкций ORBIS32.
% б.	Поддержка исключений.
% в.	Трансляция адресов: TLB и (MMU).
% г.	Периферийные устройства: Tick timer facility.
% д.	Периферийные устройства: PIC.
% е.	Модель кэша данных.
% ж.	Набор инструкций ORFPX32.
% Проверка работоспособности модели с помощью юнит-тестирования и микро-приложений, собранных с помощью GCC-or1k.
