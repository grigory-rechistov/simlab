\chapter{Знакомство с Simics}\label{chap:lab01}

\section{Цель занятия}

Ознакомиться с базовыми операциями, которые можно выполнять в рабочем окружении (\abbr workspace) симулятора Simics.

\begin{itemize*}
    \item Доступ на удалённую Unix систему с помощью SSH/VNC.
    \item Запуск симулятора. Элементы интерфейса.
    \item Сценарии работы. Гостевые системы. 
    \item Процесс загрузки ОС Linux внутри симулятора.
    \item Базовые операции: остановка модели, инспектирование состояния, сохранение на диск.
\end{itemize*}

\section{Ход работы}

\subsection{Доступ на удалённую систему через SSH}\label{subsec:ssh}

SSH (\abbr Secure SHell --- <<безопасная оболочка>>)~\cite{openssh} --- сетевой протокол прикладного уровня, позволяющий производить удаленное управление операционной системой и туннелирование TCP-соединений (например, для передачи файлов). SSH-клиенты и SSH-серверы доступны для большинства сетевых операционных систем.

Ниже будут описаны способы использования SSH из-под операционных систем Linux и Windows.

\subsubsection{Linux}

Откройте терминал и наберите команду, заменив слово \texttt{user} на свой логин:

\begin{lstlisting}
$ ssh user@cluster.iscalare.mipt.ru
\end{lstlisting}

После этого система спросит у вас пароль. Введите его. Внимание! Вводимые символы не отображаются на экране. При удачном вводе пароля вы увидите сообщения приветствия командной строки (рис.~\ref{fig:ssh-login}).

\subsubsection{Windows}

Для того чтобы подключаться к удаленному серверу из системы Windows, необходимо воспользоваться SSH-клиентом, например, PuTTY~\cite{putty} (мы рекомендуем использовать именно его):

\begin{enumerate*}
    \item Запустите PuTTY.
    \item В главном окне программы в поле \textbf{Host Name (or IP address)} введите строку \texttt{user@cluster.iscalare.mipt.ru}, заменив \texttt{user} на свой логин (рис.~\ref{fig:putty}).
    \item Порт оставьте равным 22.
    \item Сохраните сессию под некоторым именем, чтобы избежать необходимости повторного ввода в дальнейшем.
    \item После открытия окна вам будет предложено ввести пароль. Введите его и нажмите \textbf{Enter}. Внимание: вводимые символы не отображаются на экране!
    \item При удачном входе вы получите приглашение командной строки (рис.~\ref{fig:ssh-login}).
\end{enumerate*}

\begin{figure}[htb]
    \centering
    \includegraphics[width=0.6\textwidth]{./ssh-login}
    \caption{Открытие сессии SSH}
    \label{fig:ssh-login}
\end{figure}

\begin{figure}[htb]
    \centering
    \includegraphics[width=0.6\textwidth]{./putty.png}
    \caption{Окно PuTTY}
    \label{fig:putty}
\end{figure}

\subsection{Создание VNC-сессии}

VNC (\abbr Virtual Network Computing)~\cite{vnc} --- система удалённого доступа к рабочему столу компьютера, использующая протокол RFB (\abbr Remote Frame Buffer). VNC состоит из двух частей: клиента и сервера. Сервер — программа, предоставляющая доступ к экрану компьютера, на котором она запущена. Клиент (\abbr viewer) --- программа, получающая изображение экрана с сервера и взаимодействующая с ним по протоколу RFB. Управление осуществляется путём передачи нажатий клавиш на клавиатуре и движений мыши с одного компьютера на другой и ретрансляции содержимого экрана через компьютерную сеть. Система VNC платформонезависима: VNC-клиент, запущенный на одной операционной системе, может подключаться к VNC-серверу, работающему на любой другой ОС. Существуют реализации клиентской и серверной части практически для всех операционных систем, в том числе и для Java (включая мобильную платформу J2ME). К одному VNC-серверу одновременно могут подключаться множественные клиенты. Наиболее популярные способы использования VNC — удалённая техническая поддержка и доступ к рабочему компьютеру из дома.

\subsubsection{Использование VNC для доступа к удаленному серверу}

Для создани VNC-сессии необходимо выполнить следующую последовательность действий:

\begin{enumerate*}
    \item Зайдите на удалённый сервер, как описано в секции~\ref{subsec:ssh}.
    \item Запустите команду для создания графического сервера с необходимым вам разрешением экрана:
\begin{lstlisting}
$ vncserver -geometry 1280x1024
\end{lstlisting}
Рекомендуется выбирать разрешение, равное или меньшее физического разрешения экрана вашей системы.
    \item При необходимости (при первом запуске) задайте пароль (он не обязан совпадать с SSH-паролем). Пароль можно изменить впоследствии с помощью команды \texttt{vncpasswd}. Если вы забыли его, то удалить его можно, удалив файл \texttt{\$HOME/.vnc/passwd}.
    \item Запомните номер сессии (число после двоеточия в выводе команды). Например, из вывода команды ниже видно, что только что созданный номер сессии равен 4.
\begin{lstlisting}
$ vncserver
New 'mipt-head.tornado:4 (user)' desktop is mipt-head.tornado:4
Starting applications specified in /home/user/.vnc/xstartup
Log file is /home/user/.vnc/mipt-head.tornado:4.log
\end{lstlisting}
\textbf{Примечание.} Допускается иметь более одной VNC-сессии, при этом каждая будет иметь свой идентификатор (номер порта после двоеточия). Однако имейте в виду, что каждая из них потребляет небольшое количество общих системных ресурсов.
Файл журнала соответствующей сессии, например \texttt{mipt-head.tornado:4.log}, полезен при анализе проблем при запуске, например, когда открываемый впоследствии графический экран пустой или VNC-сессия вообще не запустилась.
\end{enumerate*}

\subsubsection{Замечание о необходимости тунеллирования VNC}

В настоящее время на \texttt{cluster.iscalare.mipt.ru} закрыты все внешние порты, кроме SSH (22). Поэтому для VNC-соединения, использующего порты в диапазоне 5901--5999, необходимо создавать т.н. \textbf{SSH-туннель}. Теоретически для локального конца туннеля можно выбирать любой порт (из непривилегированного диапазона 1001--65535), однако для избежания путаницы рекомендуется выбирать его равным номеру удалённого конца; в последующих примерах мы придерживаемся этого правила.

\subsubsection{Отключение VNC-сессии}

Обычно закрывать сессию VNC не требуется --- достаточно отключиться от неё. Затем можно возобновить работу, используя VNC-клиент. При необходимости уничтожение десктопа и всех запущенных приложений может быть выполнено командой

\begin{lstlisting}
$ vncserver -kill :4
\end{lstlisting}

Если необходимо уничтожить все свои VNC-сессии, используйте команду

\begin{lstlisting}
$ killall Xvnc
\end{lstlisting}

\subsubsection{Linux}

Для начала нам необходимо установить VNC-клиент на наш рабочий компьютер (если он еще не установлен). Для установки клиента на персональную ЭВМ необходимо воспользоваться пакетным менеджером. Пример для Ubuntu/Debian:

\begin{lstlisting}
$ sudo apt-get install xtightvncviewer
\end{lstlisting}

Кроме того, допускается использовать любой клиент, поддерживающий протокол VNC: RealVNC, TighVNC, Remmina, Vinagre, Krdp\dots

Затем используйте следующие опции \textbf{ssh} для создания туннеля (в примере ниже  --- для порта 5904, имени пользователя \texttt{user})

\begin{lstlisting}
$ ssh -L 5904:127.0.0.1:5904  -l user cluster.iscalare.mipt.ru
\end{lstlisting}

Если вы используете TightVNC, то для запуска VNC-сессии вам необходимо воспользоваться командой, указав после двоеточия номер вашей VNC-сессии (в данном примере номер VNC-сессии равен 4).

\begin{lstlisting}
$ vncviewer localhost:4
\end{lstlisting}

После этого система спросит у вас пароль. Введите его. Внимание! Вводимые символы не отображаются на экране. При удачном вводе пароля откроется окно VNC-сессии, готовое к работе. 

\subsubsection{Windows}

Для Windows мы рекомендуем использовать RealVNC. Инструкцию по установке и дополнительную информацию можно найти в~\cite{RealVNC}.

Допускается использовать любой клиент, поддерживающий протокол VNC: RealVNC, TighVNC, UltraVNC\dots

При открытии SSH из Putty дополнительно на складке Tunnels создаём проброс порта вашей VNC-сессии. В данном примере (рис.~\ref{fig:ssh-tunnel}) это экран VNC :4, что соответствует TCP-порту 5904.

\begin{figure}[htb]
    \centering
    \includegraphics[width=0.6\textwidth]{./ssh-tunnel.png}
    \caption{Создание ssh туннеля с помощью PuTTY}
    \label{fig:ssh-tunnel}
\end{figure}

Для подключения к удаленному рабочему столу необходимо воспользоваться VNC-клиентом (в нашем примере это RealVNC). Выполните следующую последовательность действий:

\begin{enumerate*}
    \item Откройте VNC viewer.
    \item В поле VNC Server укажите \textbf{localhost:4} (рис.~\ref{fig:realvnc}), заменив число 4 после двоеточия на номер вашей VNC-сессии.

\begin{figure}[htb]
    \centering
    \includegraphics[width=0.5\textwidth]{./realvnc.png}
    \caption{Окно RealVNC}
    \label{fig:realvnc}
\end{figure}

    \item Нажмите \textbf{Connect}.
    \item Система запросит ваш пароль от VNC-сессии. Введите его.
    \item При удачном вводе пароля откроется окно VNC-сессии готовое к работе.
\end{enumerate*}

\subsection{Базовые команды Linux}

В этой секции перечислены команды Linux, которые понадобятся для выполнения данной лабораторной работы:

\begin{itemize*}
    \item Команда \textbf{man} предназначена для формирования и  вывода справочной информации о команде \texttt{<command>}:
\begin{lstlisting}
$ man <command>
\end{lstlisting}

    \item Команда \texttt{ls} выводит информацию о \texttt{<file>} (по умолчанию выводится информация о текущей директории).
\begin{lstlisting}
$ ls <file>
\end{lstlisting}

    \item Команда \texttt{cd} меняет текущую рабочую директорию на \texttt{<directory>}.
\begin{lstlisting}
$ cd <directory>
\end{lstlisting}

    \item Команда \texttt{mkdir} создает директорию \texttt{<directory>} в текущей, если она до этого не существовала.
\begin{lstlisting}
$ mkdir <directory>
\end{lstlisting}

    \item Команда \texttt{cat} направляет \texttt{<file>} на стандартный вывод. Чаще всего это означает распечатывание его содержимого на экране.
\begin{lstlisting}
$ cat <file>
\end{lstlisting}

    \item Команда \texttt{lspci} показывает список всех PCI-устройств.
    \item Команда \texttt{mount [node] <folder>} монтирует файловую систему; вызванная без аргументов, она перечисляет уже смонтированные системы.
    \item Команда \texttt{umount <folder>} предназначена для отмонтирования файловой системы.
\end{itemize*}

Больше информации о командах Linux вы можете найти в мануалах, воспользовавшись командой \textbf{man}, описанной выше.

\section{Запуск симулятора Simics. Элементы интерфейса}

Wind River\textregistered\ Simics~\cite{getting-started} --- это быстрый, функционально точный полноплатформенный симулятор. Simics позволяет создать виртуальное окружение, в котором любая электронная система, начиная с одной платы и заканчивая целыми многопроцессорными и многоядерными системами, может быть определена, разработана и запущена.

\subsection{Создание и обновление Simics workspace}

Инсталляция Simics может быть использована совместно множеством пользователей и предпочтительно должна быть доступна только для чтения. Рабочее окружение (далее <<workspace>>) --- это персональная <<копия>> установки, в которой Simics хранит ваши собственные настройки среды и моделей и в которой вы можете держать ваши рабочие файлы. Каждый пользователь может иметь более одного workspace, и все они могут использоваться одновременно и независимо друг от друга.

\subsubsection{Создание workspace}

Для создания нового workspace вам необходимо выполнить следующую последовательность действий:

\begin{enumerate*}
    \item Создайте директорию, в которой будет находиться ваш персональный workspace (рекомендуется использовать локации внутри домашней директории пользователя) и перейдите в неё:

\begin{lstlisting}
$ mkdir <folder_name>
$ cd <folder_name>
\end{lstlisting}
    \item Вызовите команду
\begin{lstlisting}
$ /share/simics/simics-4.6/simics-4.6.32/bin/workspace-setup
Workspace created successfully
\end{lstlisting}
\end{enumerate*}

Новый workspace создан! Можно приступать к работе с симулятором.

\subsection{Запуск Simics}

После создания workspace вы можете начинать работу с Simics. Запустить симулятор вы можете с помощью команды:

\begin{lstlisting}
./simics
\end{lstlisting}

Должно открыться окно управления \textbf{Simics Control} (рис.~\ref{fig:simics-control}). Оно включает в себя иконки панели инструментов и меню, позволяющее контролировать состояния симулятора и текущей гостевой системы. Для выполнения данных лабораторных работ нам понадобится дополнительное окно с командной строкой \textbf{Simics Command Line} (рис.~\ref{fig:simics-command-line}). Выберите \textbf{Tools $\to$ Command Line Window} для того, чтобы открыть его.

\begin{figure}[ht]
    \centering
    \includegraphics[width=0.6\textwidth]{./simics-control.png}
    \caption{Окно Simics Control}
    \label{fig:simics-control}
\end{figure}

\textbf{Simics Command Line} позволяет вам управлять симуляцией с помощью ввода текстовых команд. Во время своей работы Simics будет выводить в него диагностическую информацию: события симулятора, предупреждения, сообщения об ошибках. Всё, что вы можете сделать с помощью окна \textbf{Simics Control}, вы также можете сделать и с помощью командной строки. Большинство действий в дальнейшем будет производиться именно в ней.

\begin{figure}[ht]
    \centering
    \includegraphics[width=0.6\textwidth]{./simics-command-line.png}
    \caption{Командная строка Simcs}
    \label{fig:simics-command-line}
\end{figure}

В данной работе изучается система (вычислительная платформа), основанная на процессоре Intel\textsuperscript{\textregistered} Core\textsuperscript{\texttrademark} i7. Linux используется в качестве операционной системы. В качестве среды пользователя выступает пакет под названием BusyBox~\cite{BusyBox}, часто используемый во встраиваемых (\abbr embedded) системах.

Для подключения модели воспользуйтесь окном \textbf{Simics Command Line} и введите команду:

\begin{lstlisting}
simics> run-command-file targets/x86-x58-ich10/viper-firststeps.simics
\end{lstlisting}

То же самое можно было сделать с помощью окна \textit{Simics Control}, выбрав \textbf{File $\to$ New session from script} и открыв файл \texttt{viper-firststeps.simics}, лежащий в директории \texttt{x86-x58-ich10}.

Спустя некоторое время окно \textit{Simics Control} покажет вам суммарную информацию о симулируемой системе, которую вы только что загрузили (рис.~\ref{fig:viper-control}). Также должно появиться новое окно \textbf{Serial Console on viper.mb.sb.com[0]} (рис.~\ref{fig:viper-serial-console}).

\begin{figure}[ht]
    \centering
    \includegraphics[width=0.6\textwidth]{./viper-control.png}
    \caption{Окно Simics Control после загрузки модели}
    \label{fig:viper-control}
\end{figure}

\begin{figure}[ht]
    \centering
    \includegraphics[width=0.6\textwidth]{./viper-serial-console.png}
    \caption[Окно Serial Console]{Окно \textbf{Serial Console on viper.mb.sb.com[0]}}
    \label{fig:viper-serial-console}
\end{figure}

Это новое окно является частью симуляции. Оно соединено с последовательным портом моделируемой материнской платы. Вывод сообщений гостевого ПО будет отображаться в нём. Кроме того, через него вы сможете взаимодействовать с моделируемым программным обеспечением, тогда как \textbf{Simics Command Line} используется для управления симуляцией.

\subsection{Запуск моделирования}

Теперь Simics ожидает ваших команд, вводимых через панель инструментов или через командную строку. Вы можете запустить симуляцию с помощью нажатия $\triangleright$ \textbf{Run Forward} на панели инструментов или ввода команды \textbf{continue} или \textbf{c} в командной строке.

Рассмотрим следующий пример:

\begin{lstlisting}
simcs> continue
... the simulation is running ...
running> stop
[viper.mb.cpu0.core[0][0]] cs:0xffffffff8100944d p:0x00100944d  MWait state
... the simulation is now stopped ...
simics>
\end{lstlisting}

После запуска виртуальное время начинает увеличиваться, гостевые инструкции начинают исполняться, и сообщения от программного обеспечения выводятся в окно консоли. Если вы позволите модели поработать некоторое время, то увидите сообщения загрузки в окне \textbf{Serial Console on viper.mb.sb.com[0]} (рис.~\ref{fig:booted-linux}). Симуляция может быть остановлена с помощью команды \texttt{stop} или нажатия \textbf{Stop} на панели инструментов.

\begin{figure}
	\centering
	\includegraphics[width=0.6\textwidth]{./booted-linux.png}
	\caption{Загруженная операционная система Linux}
	\label{fig:booted-linux}
\end{figure}

После загрузки ядра в гостевой консоли будет выведено приглашение для логина пользователя:
\begin{lstlisting}
busybox login:
\end{lstlisting}

Введите в него \texttt{root}, нажмите \textbf{Enter}. Теперь операционная система загружена, и вы можете начать взаимодействовать с ней, например, вводить команды в командной строке.

\paragraph{Получить листинг корневой директории}
\begin{lstlisting}
# ls /
bin      etc      host     linuxrc  root     sys
dev      home     init     proc     sbin     var
\end{lstlisting}

\paragraph{Увидеть информацию о моделируемом процессоре}
\begin{lstlisting}
# cat /proc/cpuinfo 
processor       : 0
vendor_id       : GenuineIntel
cpu family      : 6
model           : 10
model name      : Intel(R) Core(TM) i7 CPU              @ 2.00GHz
stepping        : 8
cpu MHz         : 1999.991
cache size      : 8192 KB
physical id     : 0
siblings        : 1
core id         : 2
cpu cores       : 
\end{lstlisting}

Для того чтобы закончить работу симулятора, необходимо выполнить команду \texttt{quit}, или \texttt{q}, или \texttt{exit}.

\section{Задания}

\begin{enumerate*}
    \item Загрузить модель Viper с операционной системой BusyBox~\cite{BusyBox}. Для этого необходимо из вашего workspace выполнить команду
\begin{lstlisting}
./simics targets/x86-x58-ich10/viper-busybox.simics
\end{lstlisting}
    \item Открыть окно \textbf{Simics Command Line} и запустить симуляцию.
    \item Сравнить содержимое файла \texttt{/proc/cpuinfo} симулируемой и реальной машины.
    \item Найти способы определения факта нахождения внутри симулятора (имена устройств будут выдавать факт виртуализации).
\end{enumerate*}

\iftoggle{webpaper}{
    \printbibliography[title={Список литературы к занятию}]
}{}